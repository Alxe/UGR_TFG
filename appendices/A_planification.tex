\chapter{Planificación del Proyecto}\label{app:planification}

Aunque no se ha llevado ninguna metodología concreta como podría ser, por ejemplo, \textit{Scrum}, se aplicado una metodología ágil e incremental al realizarse prototipos o bocetos sobre los que se ha continuado el desarrollo, también con gran cantidad de realimentación al solicitar opinión a mi tutor, José Samos Jiménez, tanto por correo como en horario de tutorías.

Dicho esto y considerando que, en España, un crédito \textit{ECTS} equivale a 25 horas de estudio, y que la asignatura de \textsc{Trabajo de Fin de Grado} se compone de 12 créditos \textit{ECTS}, el tiempo a dedicar a este trabajo debería haber sido 240 horas, el cual se ha desarrollado por fases descritas en \autoref{tab:planification}. Este tiempo ha sido alcanzado e incluso superado, por dedicarle horas en las que no conseguía un rendimiento muy alto. 

Este trabajo se ha desglosado por actividades en las proporciones que se muestran en \autoref{tab:planification_tasks}.

\begin{table}[htbp]
	\centering
	\begin{tabular}{r || l | c  c }
		\textbf{Período} & \textbf{Enfoque} & \textbf{Tiempo} & \textbf{Dedicación} \\
		\hline\hline
		\textbf{Primer Semestre} & Estudio e Investigación & 36 horas & 15\% \\
		\textbf{Segundo Semestre} & Experimentar y Codificar & 72 horas & 30\%\\
		\textbf{Período de Verano} & Desarrollo del Documento & 132 horas & 55\%\\
		\hline
		& Trabajo de Fin de Grado & 240 horas & 100\%
	\end{tabular}
	\caption[Planificación general del proyecto]{El esfuerzo se ha repartido a lo largo del periodo académico comprendido entre los años dos mil dieciocho y diecinueve (\textsc{2018-2019}), en una proporción irregular, por cuestiones de tiempo disponible.}
	\label{tab:planification}
\end{table}

\begin{table}[htbp]
	\centering
	\begin{tabular}{ r ||l | c c}
		\textbf{Período} & \textbf{Actividad} & \textbf{Tiempo} & \textbf{Dedicación} \\
		\hline\hline
		\textbf{I} & Estudio de Herramientas & 18 horas & 7.5\% \\
		\textbf{I} & Análisis de Viabilidad & 18 horas & 7.5\% \\
		\textbf{II} & Diseño del Software & 36 horas & 15\% \\
		\textbf{II} & Codificación del Software & 36 horas & 15\% \\
		\textbf{III} & Desarrollo del Documento & 120 horas & 50\% \\
		\textbf{III} & Incorporación en otras Soluciones & 6 horas & 2.5\% \\
		\textbf{III} & Creación de Recursos Gráficos & 6 horas & 2.5\%\\
		\hline
		 & Trabajo de Fin de Grado & 240 horas & 100\%
	\end{tabular}
	\caption[Planificación por actividad]{El desarrollo de cada actividad ha estado realmente intercalado con otras actividades, por ejemplo la creación de recursos gráficos se ha llevado a cabo durante la realización del documento. Los tiempos y porcentajes dados son estimaciones de la realidad.}
	\label{tab:planification_tasks}
\end{table}

\section*{Explicación}
Elegí desarrollar y exponer este trabajo en un año, académicamente, muy pesado al estar cursando once asignaturas además de este trabajo, completando un total de 78 créditos \textit{ECTS}.

Como tal, el proyecto partió de una planificación rígida a algo más flexible. Quise partir de repartir mi tiempo a partes iguales en investigar, desarrollar y escribir. Esto, como cabía esperar, no se cumplió y acabó siendo diferente, véase \autoref{tab:planification}.

Como tal, empecé en el primer semestre del año académico buscando ideas sobre qué desarrollar como objeto de mi \textsc{Trabajo de Fin de Grado}. En la susodicha asignatura de \textsc{Sistemas de Información Geográficos} elegí el tema a tratar, por mi cariño a la asignatura y mi interés en una tecnología libre como \textit{GeoPackage}.

Ese semestre no logré mucho más que aprender sobre \textit{GIS} y diseñar ciertos prototipos, como la inyección de funciones \textit{SQL} para imitar a \textit{pgRouting}, pero por limitaciones de \textit{SQLite}, tuve que desestimarlo.

Al siguiente, segundo semestre, logré desarrollar más el texto, particularmente la estructura y los elementos a introducir, pero la ausencia de tiempo me hizo dejar la entrega final a Septiembre.

En los meses de verano, ya con todo o gran parte del código desarrollado y la idea bien clara, he dedicado gran parte de mi tiempo diario a el desarrollo del documento, sea texto o recursos como figuras o códigos.

Consolidando toda esta información explicada arriba, podríamos decir que este proyecto se ha desarrollado en tres fases, primer y segundo semestre académico y los meses de verano, cada uno con un enfoque y una dedicación concreta.