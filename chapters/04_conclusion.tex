\chapter{Conclusión y Trabajo Futuro}
\label{ch:conclusion}

Como se ha podido comprobar en el capítulo anterior, es muy viable la creación de un \textit{GeoPackage} con información topológica para realizar análisis de redes de carreteras u otro tipo. 

Esto no quiere decir que sea la única ni mejor solución, si no una más, para ámbitos en los que otras soluciones más maduras no puedan implementarse.
De igual manera, sigue siendo una posibilidad más que válida, con un nicho de mercado muy interesante como pueda ser el encaminamiento en local o el empaquetado y envío de datos geográficos enrutables.

Así pues, a continuación se presentan dos secciones, una para desglosar qué objetivos se han cumplido y cómo se ha llevado esto a cabo, y otra para definir qué trabajo futuro podría realizarse sobre las conclusiones resultantes de este análisis.

\section{Objetivos}

Tanto el objetivo principal como los objetivos secundarios han sido completados a lo largo del desarrollo del \textsc{Trabajo de Fin de Grado}. Estos objetivos son listados en la \autoref{sec:objectives}.

El objetivo principal se ha cumplido en toda la realización del \autoref{ch:geopackage_routing}, llevándose a cabo el análisis que se esperaba, obteniendo como conclusiones que sí es viable y las pautas a seguir para obtener un \textit{GeoPackage} enrutable, así como la estructura interna que podrían seguir los datos y algoritmos o extractos de código fuente para completar el proceso.

Los objetivos secundarios se han completado de igual manera, en el proceso de documentación y realización previa a la redacción del documento, y esto se puede observar en el texto y herramientas desarrolladas, a disposición en este mismo documento o en los recursos adicionales presentados al final del mismo.

\begin{itemize}
	\setlength\itemsep{0em}
	\item Se ha conocido la estructura de la base de datos de \textit{OpenStreetMap} a partir de la \textit{OsmWiki}\footnote{Véase \url{https://wiki.openstreetmap.org/wiki/}} y se ha aprendido cómo usar este a partir de la extracción de datos a partir de las herramientas del proyecto \textit{Osmium}.
	
	\item Se han identificado más tipos de contenedores de datos geográficos a través de recursos como sean \autocite{volaya} o \autocite{bolstad} y qué ventajas o desventajas presentan.
	
	\item Se ha encontrado herramientas como \textit{Osmium}, \textit{GeoPandas}, \textit{NetworkX}, entre otras, para la creación, manipulación y visualización de datos geográficos en un entorno general.
	
	\item Referente al punto anterior, muchas herramientas anteriormente citadas funcionan bajo un entorno \textsc{Python}, o tienen algún tipo de \textit{binding}, así pues se exhibe que dicho entorno es muy capaz a la hora de trabajar con información geográfica y análisis estadísticos o de otro tipo.
\end{itemize}

\section{Trabajo Futuro}
El trabajo, una vez finalizado y con sus conclusiones presentadas, podría llegar a extenderse por varios caminos.

Por un lado, se podría ver la posibilidad de la estandarización del modelo de representación de componentes topológicas, la propia red de conexiones entre nodos y los nodos en sí, como una extensión del estándar \textit{GeoPackage}. De esto llevarse a cabo, habría que producir un boceto técnico y presentarse al \textit{Open Geospatial Consortium} como una extensión de la comunidad\footnote{Véase \url{https://www.geopackage.org/extensions.html}}. Una vez esto, debería ser publicitado, implementado y revisado reiteradamente hasta que finalmente acepte la madurez necesaria para que se convierta en una extensión oficial.

Por otro lado, cabe la posibilidad de extender la prueba de concepto desarrollada, añadiendo mayor cantidad de funcionalidades aunque dependa internamente de los mismos proyectos ya comentados, llámese \textit{NetworkX}, \textit{GeoPandas}, \textit{Osmium}, además de otros. El objetivo principal de este proyecto debería ser la accesibilidad a desarrolladores, así pues siendo un nexo de enlace entre las muchas herramientas ya comentadas.

Otra posibilidad relacionada con la anterior, sería utilizar las herramientas de más bajo nivel, fuera del ecosistema \textsc{Python}, y buscar herramientas más portables a otras plataformas, como pueda ser basadas en tecnologías como \textsc{C++} o \textsc{Java}, idealmente con el mismo planteamiento que la propuesta anterior, la accesibilidad a otros desarrolladores.

Este trabajo también puede, propiamente dicho, ayudarme a encontrar trabajo en el sector \textit{GIS}, donde podría plasmar todo el conocimiento adquirido y refinado en la elaboración del análisis y de la misma implementación.