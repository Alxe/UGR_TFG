\begin{minipage}[c]{0.5275\textwidth}
\itshape
\chapter*{\itshape Agradecimientos}
	Quiero agradecer este trabajo, culmen de mi estancia en la Universidad de Granada, tanto a mis familiares más cercanos, mi madre y mi hermano, por estar siempre ahí y apoyarme cuando lo necesitaba, como a todos los amigos que he hecho durante estos años en la Universidad por hacer de estos una experiencia inolvidable y a todos mis profesores, indistintamente de mi opinión personal, por compartir conmigo sus conocimientos, resolver mis dudas y hacerme crecer como estudiante, con mención de honor a mi tutor por su tiempo y ayuda con su supervisión.
	También, una sincera muestra de gratitud a cada uno de los lectores del documento, pues hubiese sido una gran pena que este esfuerzo cayese en el olvido.\par
\end{minipage}

\blankpage

%% Español:
\begin{poliabstract}{Resumen}
	El objetivo de este proyecto consiste en llevar a cabo un análisis de viabilidad y desarrollo de una prueba de concepto para dotar de características de encaminamiento al estándar \textit{GeoPackage}, un contenedor de datos geográficos. 
	
	Para lograr dicha tarea, se ha utilizado herramientas de software libre, así como el desarrollo de una pequeña biblioteca que, dado una estructura similar a un grafo de nodos y enlaces, gestiona la carga, tratamiento y manipulación de datos geográficos para obtener un camino entre dos puntos.

\paragraph*{Palabras Clave}
	encaminamiento, geopackage, información geográfica, qgis, pgrouting, gis, networkx, python, openstreetmap, geopandas\par
\end{poliabstract}

%% Inglés:
\begin{poliabstract}{Abstract}
	The goal of this project is to analyze the viability and to develop a proof-of-concept system of adding routing capabilities to the \textit{GeoPackage} standard, a geospatial data container. 
	
	To achieve said goal, a variety of free software tools were used, as well as the development of a small library that, given a graph-like structure of nodes and links, is able to load, process and manipulate geospatial data to obtain a route between two points.

\paragraph*{Keywords}
	routing, geopackage, geospatial information, qgis, pgrouting, gis, networkx, python, openstreetmap, geopandas\par
\end{poliabstract}

\blankpage
